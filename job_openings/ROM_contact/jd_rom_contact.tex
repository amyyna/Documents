\documentclass[12pt]{article}

% Packages
\usepackage{geometry}      % Page margins
\usepackage{graphicx}      % For including images
\usepackage{amsmath}       % For mathematical symbols and equations
\usepackage{hyperref}      % For links and clickable table of contents
\usepackage{longtable}     % For long tables
\usepackage{fancyhdr}      % For customizing headers and footers
\usepackage{xcolor}

% Page Setup
\geometry{top=1in, bottom=1in, left=1in, right=1in}
\pagestyle{fancy}
\fancyhead[L]{Internship Proposal}
\fancyhead[R]{College of Computing - UM6P}

\title{Model reduction for constrained problems }
\author{
	College of Computing, University Mohammed 6 Polytechnic \\
}
\date{\today}

\begin{document}
	
	\maketitle
	
	\section{Introduction}
	In many engineering and physics problems, particularly in nonlinear and multiscale simulations, the size of the problem can make direct numerical solutions intractable. By reducing the number of degrees of freedom in the system—while maintaining accuracy—model reduction allows for more efficient and scalable simulations. 
	Model reduction techniques, such as reduced basis methods or proper orthogonal decomposition (POD), can be used to reduce the dimensionality of numerical simulations of physical systems, making it possible to perform faster simulations while preserving the accuracy of the material response. This approach is particularly useful in structural and civil engineering, where efficiency and speed are critical.
	
	The goal of this PhD thesis is to explore how model reduction techniques can be applied to solve complex Partial Differential Equations (PDEs) with inequality constraints.
	A first example of such constraints are the Lorentz cone-type constraints in problems of frictional contact, and a second one would be elastoplasticity. 
	The adopted numerical approach is twofold. First, we will address the reduction of the model problem and the construction of a suitable reduced basis space while focusing only on the nonlinear constraint. 
	Second, we will address any further non-linearities to generalize the solution.
	
	At a later-time, we would explore a combination of numerical methods (model reduction) and statistical methods (machine/deep learning)  to leverage existing data while keeping physical properties of the computed solutions.
	
	Other closely connected problems can be addressed as opportunities arise through the PhD candidate's research journey.
	
	
	\section{Milestones}
	The PhD will ideally consist of the following major milestones :
	\begin{enumerate}
		\item Get familiarized with model reduction of PDEs and Lorentz cone type constraints.
		\item Implement a proof-of-concept using Python.
		\item Extend to elastoplacity.
		\item Work on devising a combined numerical-statistical approach.
		\item Assess it via test cases and/or theoretical error bounds.
	\end{enumerate}
	
	
	\section{Qualifications}
	%This internship is a research oriented end of studies project. It is better suited for students interested in pursuing a research path after completion of their degree. The intended start date is early 2025.
	Ideal candidates should have the following qualifications :
	\begin{itemize}
		\item Currently completing a masters' or engineering degree in applied mathematics, mechanical engineering, numerical simulation or a related field.
		\item A solid understanding of some (preferably many) of the following fields : optimization, dimensionality reduction, techniques, partial differential equations, numerical methods, machine learning/deep learning.
		\item Good knowledge and previous use of Python for academic projects or internships.
		\item A solid understanding of finite elements is preferred, but not mandatory.
	\end{itemize}
	\section{How to apply}
	%	\begin{itemize}
		%	\item Amina Benaceur : amina.benaceur@um6p.ma
		%	\item Mohammed-Khalil Ferradi : khalil.ferradi@um6p.ma
		%	\end{itemize}
	In your application, include a CV, your college transcripts, and a couple of potential references%, a few lines explaining your interest in the internship, and your earliest starting date.
	%The email title should preferably use the format [ROM-Internship:FirstName-LastName-StartDate].
	
	%\textcolor{white}{\cite{niakh2023reduced, benaceur2020reduced}}
	%\bibliographystyle{plain} 
	%\bibliography{biblio}   
	
\end{document}
