\documentclass[twocolumn,12pt,a4paper]{article}
\usepackage[latin1]{inputenc}
\usepackage[T1]{fontenc} 
\usepackage[english]{babel} 

\pagestyle{empty} 


\usepackage{amsmath, amsthm}
\newtheorem{theorem}{Theorem}
\newtheorem{proposition}{Proposition}
\newtheorem{lemma}{Lemma}
\newtheorem*{proof*}{Proof}
\newtheorem{definition}{Definition}

\newtheorem{exercise}{Exercise}
\newtheorem{question}{Question}

\newtheorem{example}{Example}

\newtheorem{remark}{Remark}

\usepackage[a4paper,left=.8cm, right=.8cm,top=1.5cm,bottom=1.5cm]{geometry}
\setlength{\columnsep}{1.2cm}

\usepackage{cancel}
\usepackage{bm}
\usepackage{amssymb,amsfonts}
\usepackage{mathrsfs}
\usepackage{color}
%\usepackage{hyperref}
\usepackage{dsfont}
\usepackage{graphicx}
\usepackage{algorithmicx}
\usepackage[ruled]{algorithm}
\usepackage{algpseudocode}
\usepackage{marginnote}
\newcommand{\Ptr}{\mathcal P^{\rm tr}}
\newcommand{\tr}{{\rm tr}}
\newcommand{\N}{\mathbb N}
\newcommand{\calN}{\mathcal N}
\newcommand{\bP}{\bold P}
\newcommand{\calK}{\mathcal K}
\newcommand{\calF}{\mathcal F}
\newcommand{\calH}{\mathcal H}
\newcommand{\calP}{\mathcal P}
\newcommand{\calC}{\mathcal C}
\newcommand\red[1]{\textcolor{red} {#1} }
\newcommand{\R}{\mathbb R}
\newcommand{\bX}{\bar X}
\newcommand{\Ktr}{\calK^{\rm tr}}
\title{ \bfseries \Huge {The Bernoulli process}}    
\vspace{-4cm}        

\author{Amina Benaceur}       
\vspace{-4cm}        
\newcounter{num}  % Create a new counter for paragraphs
\begin{document}
	\maketitle
	\setcounter{num}{1}  % Start the paragraph counter at 1
	
	\thispagestyle{empty} 
	\paragraph{Brief recap}
	
	\begin{enumerate}
		\item Arrival processes (probability of one sequence)
		\item We define the Bernoulli process as a sequence \( X_1, X_2, \dots \) of i.i.d. Bernoulli random variables \( X_i \) where for each \( i \)
		$$
		P(X_i = 1) = P(\text{success at the } i\text{th trial}) = p,
		$$ $$
		P(X_i = 0) = P(\text{failure at the } i\text{th trial}) = 1 - p,
		$$
		
		\item  Let \( S \) be the number of successes in \( n \) i.i.d. trials of a Bernoulli process. \( \mathbf{ S \sim Bin(n,p)}\). 
		Its PMF, mean, and variance are:
		\[
		P(S=k) = \binom{n}{k} p^k (1 - p)^{n-k}, \quad k = 0, 1, \dots, n,
		\]
		\[
		E[S] = np, \quad \text{V}[S] = np(1 - p).
		\]
		
		\item (Memorylessness) A distribution is said to be memoryless if a random variable $X$ from that distribution satisfies
		$P(X \geq s + t \mid X \geq s) = P(X \geq t)$, for all $s, t > 0$.
		\item Let \( T \) be the number of trials up to (and including) the first success in a Bernoulli process. \( \mathbf{ T \sim Geom(p)}\). 
		Its PMF, mean, and variance are:
		\[
		P(T=t) = (1 - p)^{t-1} p, \quad t = 1, 2, \dots,
		\]
		\[
		E[T] = \frac{1}{p}, \quad \text{V}[T] = \frac{1 - p}{p^2}.
		\]
		\item  	The number \( T - n \) of trials until the first success after time \( n \) has a geometric distribution with parameter \( p \), and is independent of the past.
		\[
		P(T - n = k) = (1 - p)^k p, \quad k = 0, 1, 2, \dots
		\]
		\item  For any given time \( n \), the sequence of random variables \( X_{n+1}, X_{n+2}, \dots \) (the future of the process) is also a Bernoulli process, and is independent from \( X_1, \dots, X_n \) (the past of the process).
		\item \textbf{First string of losing days}.
		\item  The PMF of arrival times $Y_k$ in a Bernoulli process is given by:
		\[
		P(Y_k = t) = \binom{t-1}{k-1} p^k (1 - p)^{t-k}, \text{ for } t \geq k,
		\]
		and 0 otherwise.
		This is known as the {\bfseries Pascal PMF} of order $k$.
	\end{enumerate}
	
	
	\textbf{Exercice:} Suppose at the CC entrance a taxi arrives every 15 minutes, and the
	probability that an arriving taxi is empty is \( \frac{1}{5} \).
	
	\begin{enumerate}
		\item Suppose you have just arrived at the entrance and are waiting for an e
		\textbf{Exercice:} Suppose at the CC entrance a taxi arrives every 15 minutes, and the
		probability that an arriving taxi is empty is \( \frac{1}{5} \).
		
		\begin{enumerate}
			\item Suppose you have just arrived at the entrance and are waiting for an empty taxi. When you arrive, you just miss an empty taxi. 
			What is the probability that you will have to wait at least 45 minutes (3 taxi arrivals) for an empty taxi?
			
			\item Suppose you have just arrived at the entrance and are waiting for an empty taxi. When you arrive, you just miss an empty taxi. 
			What is the probability that you will have to wait at least 45 minutes (3 taxi arrivals) for an empty taxi?
			
			\item Suppose you are at the entrance, and 10 taxis have arrived and left, and
			none of them was empty. What is the probability that you will
			have to wait at least 45 minutes (3 taxi arrivals) for the next empty taxi?
		\end{enumerate}mpty taxi. When you arrive, you just miss an empty taxi. 
		What is the probability that you will have to wait at least 45 minutes (3 taxi arrivals) for an empty taxi?
		
		\item Suppose you have just arrived at the entrance and are waiting for an empty taxi. When you arrive, you just miss an empty taxi. 
		What is the probability that you will have to wait at least 45 minutes (3 taxi arrivals) for an empty taxi?
		
		\item Suppose you are at the entrance, and 10 taxis have arrived and left, and
		none of them was empty. What is the probability that you will
		have to wait at least 45 minutes (3 taxi arrivals) for the next empty taxi?
	\end{enumerate}
\end{document}