\documentclass[article,12pt,a4paper]{article}
\usepackage[latin1]{inputenc}
\usepackage[T1]{fontenc} 
\usepackage[english]{babel} 

\pagestyle{empty} 


\usepackage{amsmath, amsthm}
\newtheorem{theorem}{Theorem}
\newtheorem{proposition}{Proposition}
\newtheorem{lemma}{Lemma}
\newtheorem*{proof*}{Proof}
\newtheorem{definition}{Definition}

\newtheorem{exercise}{Exercise}
\newtheorem{question}{Question}

\newtheorem{example}{Example}

\newtheorem{remark}{Remark}

\usepackage[a4paper,left=.8cm, right=.8cm,top=1.5cm,bottom=1.5cm]{geometry}
\setlength{\columnsep}{1.2cm}

%\usepackage{cancel}
\usepackage{bm}
\usepackage{amssymb,amsfonts}
\usepackage{mathrsfs}
\usepackage{color}
%\usepackage{hyperref}
\usepackage{dsfont}
\usepackage{graphicx}
\usepackage{algorithmicx}
\usepackage[ruled]{algorithm}
\usepackage{algpseudocode}
\usepackage{marginnote}
\newcommand{\Ptr}{\mathcal P^{\rm tr}}
\newcommand{\tr}{{\rm tr}}
\newcommand{\N}{\mathbb N}
\newcommand{\calN}{\mathcal N}
\newcommand{\bP}{\bold P}
\newcommand{\calK}{\mathcal K}
\newcommand{\calF}{\mathcal F}
\newcommand{\calH}{\mathcal H}
\newcommand{\calP}{\mathcal P}
\newcommand{\calC}{\mathcal C}
\newcommand{\R}{\mathbb R}
\newcommand{\bX}{\bar X}
\newcommand{\Ktr}{\calK^{\rm tr}}
\title{ \bfseries \Huge {Handout 3: Bernoulli and Poisson processes }}    
\vspace{-4cm}        

\date{Due date : February $24^{th}$}       
\vspace{-4cm}        
\newcounter{num}  % Create a new counter for paragraphs
\begin{document}
	\maketitle
	\setcounter{num}{1}  % Start the paragraph counter at 1
	
	\thispagestyle{empty} 
	\paragraph{Brief recap}
	\begin{enumerate}
		\item 
	\end{enumerate}
	
	
		
		\stepcounter{num} 
		\paragraph{Exercise \thenum.}
		Let $X_i$ ($i = 1, 2, \dots$) be i.i.d. random variables with mean 0 and variance 2; $Y_i$ ($i = 1, 2, \dots$) be i.i.d. random variables with mean 2. Assume that all variables $X_i$, $Y_j$ are independent. Consider the following statements:
		
		\begin{enumerate}
			\item[(A)] $\frac{X_1 + \dots + X_n}{n}$ converges to 0 in probability as $n \to \infty$.
			\item[(B)] $\frac{X_1^2 + \dots + X_n^2}{n}$ converges to 2 in probability as $n \to \infty$.
			\item[(C)] $\frac{X_1 Y_1 + \dots + X_n Y_n}{n}$ converges to 0 in probability as $n \to \infty$.
		\end{enumerate}
		
		Which of these statements are always true? Write \texttt{True} or \texttt{False} in each of the boxes below.
		\[
		\fbox{\parbox[c][1.5cm][t]{4cm}{\textbf{A.} \hspace{0.5cm}}} \hspace{1cm} 
		\fbox{\parbox[c][1.5cm][t]{4cm}{\textbf{B.} \hspace{0.5cm}}} \hspace{1cm} 
		\fbox{\parbox[c][1.5cm][t]{4cm}{\textbf{C.} \hspace{0.5cm}}}
		\]
		
		
		\stepcounter{num} 
		\paragraph{Exercise \thenum.}	
			\textbf{Problem 21.4.} \\
			In the fair Gambler's Ruin game with initial stake of $n$ dollars and target of $T$ dollars, let $e_n$ be the number of \$1 bets the gambler makes until the game ends (because he reaches his target or goes broke).
			
			\begin{enumerate}
				\item Describe constants $a$, $b$, $c$ such that
				\begin{equation}\label{eq:rec}
					e_n = a e_{n-1} + b e_{n-2} + c 
				\end{equation}
				for $1 < n < T$.
				 \vspace{7cm}
				\item Let $e_n$ be defined by~\eqref{eq:rec} for all $n > 1$, where $e_0 = 0$ and $e_1 = d$ for some constant $d$. Derive a closed form (involving $d$) for the generating function $E(x) = \sum_{n=0}^{\infty} e_n x^n$.
				
				\item Find a closed form (involving $d$) for $e_n$.
				
				\vspace{5cm}
				
				\item Use part (c) to solve for $d$.
				
				\vspace{5cm}
				\item Prove that $e_n = n(T - n)$.
				
				\vspace{5cm}
			\end{enumerate}
			
		
		
		
		
	
	\end{document}