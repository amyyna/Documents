\documentclass[12pt,a4paper]{article}
\usepackage[latin1]{inputenc}
\usepackage[T1]{fontenc} 
\usepackage[english]{babel} 

\pagestyle{empty} 

\usepackage{tikzsymbols}
\usepackage{textcomp}
\usepackage{parskip}
\usepackage{amsmath, amsthm}
\newtheorem{theorem}{Theorem}
\newtheorem{proposition}{Proposition}
\newtheorem{lemma}{Lemma}
\newtheorem*{proof*}{Proof}
\newtheorem{definition}{Definition}

\newtheorem{exercise}{Exercise}
\newtheorem{question}{Question}

\newtheorem{example}{Example}

\newtheorem{remark}{Remark}

\usepackage[a4paper,left=.8cm, right=.8cm,top=1.cm,bottom=1.5cm]{geometry}
\setlength{\columnsep}{1.2cm}

\usepackage{cancel}
\usepackage{bm}
\usepackage{amssymb,amsfonts}
\usepackage{mathrsfs}
\usepackage{color}
%\usepackage{hyperref}
\usepackage{dsfont}
\usepackage{graphicx}
\usepackage{algorithmicx}
\usepackage[ruled]{algorithm}
\usepackage{algpseudocode}
\usepackage{marginnote}
\newcommand{\Ptr}{\mathcal P^{\rm tr}}
\newcommand{\tr}{{\rm tr}}
\newcommand{\N}{\mathbb N}
\newcommand{\calN}{\mathcal N}
\newcommand{\bP}{\bold P}
\newcommand{\calK}{\mathcal K}
\newcommand{\calF}{\mathcal F}
\newcommand{\calH}{\mathcal H}
\newcommand{\calP}{\mathcal P}
\newcommand{\calC}{\mathcal C}
\newcommand\red[1]{\textcolor{red} {#1} }
\newcommand{\R}{\mathbb R}
\newcommand{\bX}{\bar X}
\newcommand{\Ktr}{\calK^{\rm tr}}
\title{ \bfseries \Huge {Stochastic processes : Quiz}}    
\vspace{-4cm}        

\date{\today}  

\vspace{-8cm}     

\newcounter{num}  % Create a new counter for paragraphs
\begin{document}
	\maketitle
	
	\paragraph{Question.} What makes a stochastic process Markovian? You can choose to explain either in plain english or using a mathematical formula.
	
	\vspace{6cm}   
	\paragraph{Problem.}
	The professor warns CC-CI1 students not to use computers in the classroom.
	Each time a student attempts pushing the boundaries by using a computer `again', the professor has probability $p>0$ of giving them another warning, and probability $1-p$ of excluding them from class.
	Once the professor excludes the first student from class for using a computer, the professor no longer gives warnings.
	\begin{enumerate}
		\item How do you model this problem as a Markov chain?
		\vspace{11cm} 
		
		\item Let $i$ be a transient state of the Markov chain above. 
		What is the probability of never returning to $i$ starting from $i$?
		\vspace{9cm} 
		
		\item Starting from $i$, let $X$ be the number of
		times that the chain returns to $i$ before leaving forever.
		Find the distribution of $X$.
		\vspace{13cm} 
	\end{enumerate}
	%\newpage
	
	\textcolor{white}{a}
	
	%\vspace{24cm}
	
	\fbox{
		\parbox{10cm}{
			\textcolor{white}{a}\\
			Full name*: \\
			\\
			*This is a make-up quiz due to medical absence.
		}
	}
\end{document}