\documentclass[12pt]{article}
\usepackage{amsmath}
\usepackage{graphicx}
\usepackage{enumitem}
\usepackage{hyperref}

\usepackage[a4paper,left=1cm, right=1cm,top=1.5cm,bottom=1.5cm]{geometry}
\title{\bfseries \Huge Lab session: Markov Chain Monte Carlo}
\author{}
\date{}

\begin{document}
	\maketitle
	
	\section*{Learning Objectives}
	By the end of this lab, you should be able to :
	\begin{itemize}
		\item Understand the relevance of Markov chains in sampling.
		\item Implement the Metropolis-Hastings algorithm in Python.
		\item Analyze and interpret the behavior of MCMC samplers.
		\item Apply the algorithm to both continuous and discrete probability distributions.
	\end{itemize}
	
	\section*{Part 1: Conceptual Questions}
	
	\begin{enumerate}[label=\textbf{Q\arabic*.}]
		\item Define a Markov chain and explain what is meant by the stationary distribution.
		
		\item Why do we use MCMC methods in Bayesian inference? What problems do they help to solve?
		
		\item Describe each step of the Metropolis-Hastings algorithm. What role does the proposal distribution play?
		
		\item What could happen if the proposal distribution’s variance is too large or too small?
		
		\item How would you evaluate if your Markov chain has converged?
		
		\item For the Gaussian example:
		\begin{itemize}
			\item Plot the histogram of samples and overlay the target PDF.
			\item What does the trace plot reveal about the sampler?
			\item How does changing the proposal standard deviation affect the sampling?
		\end{itemize}
		
		\item For the Zipf distribution:
		\begin{itemize}
			\item Compare the sampled histogram with the true PMF.
			\item What difficulties arise when sampling from heavy-tailed or discrete distributions?
		\end{itemize}
	\end{enumerate}
	
	\section*{Extension Questions (Optional)}
	
	\begin{enumerate}[label=\textbf{E\arabic*.}]
		\item Modify the sampler to work on a 2D target distribution (e.g., a bivariate Gaussian). What changes are necessary?
		\item Implement a different discrete proposal strategy for the Zipf sampler (e.g., geometric steps).
		\item Compare acceptance rates across different samplers. What rate seems to balance convergence and exploration?
	\end{enumerate}
	
\end{document}
