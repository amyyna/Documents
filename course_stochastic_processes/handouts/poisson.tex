\documentclass[twocolumn,12pt,a4paper]{article}
\usepackage[latin1]{inputenc}
\usepackage[T1]{fontenc} 
\usepackage[english]{babel} 

\pagestyle{empty} 


\usepackage{amsmath, amsthm}
\newtheorem{theorem}{Theorem}
\newtheorem{proposition}{Proposition}
\newtheorem{lemma}{Lemma}
\newtheorem*{proof*}{Proof}
\newtheorem{definition}{Definition}

\newtheorem{exercise}{Exercise}
\newtheorem{question}{Question}

\newtheorem{example}{Example}

\newtheorem{remark}{Remark}

\usepackage[a4paper,left=.8cm, right=.8cm,top=1.5cm,bottom=1.5cm]{geometry}
\setlength{\columnsep}{1.2cm}

\usepackage{cancel}
\usepackage{bm}
\usepackage{amssymb,amsfonts}
\usepackage{mathrsfs}
\usepackage{color}
%\usepackage{hyperref}
\usepackage{dsfont}
\usepackage{graphicx}
\usepackage{algorithmicx}
\usepackage[ruled]{algorithm}
\usepackage{algpseudocode}
\usepackage{marginnote}
\newcommand{\Ptr}{\mathcal P^{\rm tr}}
\newcommand{\tr}{{\rm tr}}
\newcommand{\N}{\mathbb N}
\newcommand{\calN}{\mathcal N}
\newcommand{\bP}{\bold P}
\newcommand{\calK}{\mathcal K}
\newcommand{\calF}{\mathcal F}
\newcommand{\calH}{\mathcal H}
\newcommand{\calP}{\mathcal P}
\newcommand{\calC}{\mathcal C}
\newcommand\red[1]{\textcolor{red} {#1} }
\newcommand{\R}{\mathbb R}
\newcommand{\bX}{\bar X}
\newcommand{\Ktr}{\calK^{\rm tr}}
\title{ \bfseries \Huge {Handout 3: Bernoulli and Poisson processes }}    
\vspace{-4cm}        

\date{Due date : February $24^{th}$}       
\vspace{-4cm}        
\newcounter{num}  % Create a new counter for paragraphs
\begin{document}
	\maketitle
	\setcounter{num}{1}  % Start the paragraph counter at 1
	
	\thispagestyle{empty} 
	\paragraph{Brief recap}
	\begin{enumerate}
			\item (Memorylessness) A distribution is said to be memoryless if a random variable $X$ from that distribution satisfies
		$P(X \geq s + t \mid X \geq s) = P(X \geq t)$, for all $s, t > 0$.
		\item  The PMF of interarrival times $Y_k$ in a Bernoulli process is given by:
		\[
		p_{Y_k}(t) = \binom{t-1}{k-1} p^k (1 - p)^{t-k}, \quad t = k, k+1, \dots,
		\]
		This is known as the Pascal PMF of order $k$.
		\item A sequence of arrivals in continuous time is called a Poisson process with rate $\lambda$ if
	\begin{enumerate}
		\item The number of arrivals in disjoint time intervals are independent.
		\item The number of arrivals in an interval of length $\tau$ is given by the Poisson distribution 
		$
		P(k,\tau) = \frac{(\lambda\tau)^k}{k!}e^{-\lambda\tau}.
		$
	\end{enumerate}	
	\item The number of arrivals $N_\tau$ over an interval of length $\tau$ in a Poisson process satisfies $E[N_\tau] = \lambda\tau$ and $ V[N_\tau] = \lambda\tau$.
	\item (Merging) Let $\{N_1(t)\}_{t > 0}$ and $\{N_2(t)\}_{t > 0}$ be independent Poisson processes with rates $\lambda_1$ and $\lambda_2$, respectively. 
	The merged process $\{N (t) = N_1(t) + N_2(t)\}_{t > 0}$ is a Poisson process with rate $\lambda_1+\lambda_2$.
	\item (Splitting) Let $\{N_1(t)\}_{t > 0}$ be a Poisson process with rate
	$\lambda$, and classify each arrival in the process as a type-1 event with probability $p$ and a type-2 event with probability $1 - p$, independently. 
	Then the type-1 events form a Poisson process with rate $\lambda p$, the type-2 events form a Poisson process with rate
	$\lambda (1-p)$, and these two processes are independent.
	\end{enumerate}




	\paragraph{Exercise \thenum.}
	$X$ is a random variable with memoryless distribution with CDF $F$ and PMF $p_i = P(X = i)$.
	Find an expression for $P(X \geq j + k)$ in terms of $F(j)$, $F(k)$, $p_j$, $p_k$.
	
	\stepcounter{num} 
	\paragraph{Exercise \thenum.}
	Let $N_t$ be the number of arrivals up until time $t$ in a Poisson process of rate $\lambda$, and let $T_n$ be the time of the $n$-th arrival. Consider statements of the form
	\[
	P(N_t  \gtrless_1 \ n) = P(T_n \gtrless_2 \ t),
	\]\\
	where $\gtrless_1$ and $\gtrless_2$ are replaced by symbols from the list $<, \leq, \geq, >$. Which of these statements are true?
	
	
	\stepcounter{num} 
	\paragraph{Exercise \thenum.}
	Passengers arrive at a bus stop according to a Poisson process with rate $\lambda$. The arrivals of buses are exactly $t$ minutes apart. 
	\begin{enumerate}
		\item 
	Show that on average, the sum of the waiting times of the riders on one of the buses is $\frac 1 2 \lambda t^2$.
	\end{enumerate}
	
	
	\stepcounter{num} 
	\paragraph{Exercise \thenum.}
	Suppose we have a couple of Poisson processes with rates $\lambda_1, \lambda_2$ respectively. 
	For an interval of length $ \delta$ small enough, give an approximation of $P((k_1,k_2);\delta)$ for $k_1,k_2\in\{1,2\}$.
	A student receives phone calls according to a Poisson process with rate $ \lambda$. 
	Unfortunately she has lost her cell phone charger. 
	The battery's remaining life is a random variable $T$ with mean $\mu$ and variance $\sigma^2$. Let $N(T)$ be the number of phone calls she receives before the battery dies.
	\begin{enumerate}
		\item Find $ {\rm E}[N(T)]$ and ${\rm V}[N(T)]$.
	\end{enumerate}
	
	
	
	\stepcounter{num} 
	\paragraph{Exercise \thenum.}
	Emails arrive in your inbox according to a Poisson process with rate $\lambda$, measured in emails per hour.
	Each email is studies-related with probability $p$ and personal with probability $1 - p$. 
	The amount of time it takes to answer a studies-related email is a random variable with mean $\mu_W$ and variance $\sigma_W^2$. 
	The amount of time it takes to answer a personal email has mean $\mu_P$ and variance $\sigma_P^2$.
	The response times for different emails are independent.
	
	What is the average amount of time you have to spend answering all the emails that arrive in a $t$-hour interval? What about the variance?
	
	
	
	\stepcounter{num} 
	\paragraph{Exercise \thenum.}
On a whatsApp question-and-answer group, $N \sim {\rm Pois}(\lambda_1)$ questions will be posted tomorrow, with $\lambda_1$ measured in questions/day. Given $N$ , the post times are i.i.d. and
uniformly distributed over the day (a day begins and ends at midnight). 
When a question is posted, it takes an ${\rm Exp}(\lambda_2)$ amount of time (in days) for an answer to be posted,
independently of what happens with other questions.
\begin{enumerate}
	\item Find the probability that a question posted at a uniformly random time tomorrow will not yet have been answered by the end of that day.
\end{enumerate}
	
	
	\stepcounter{num} 
	\paragraph{Exercise \thenum.}
	In an endless football match, goals are scored according to a Poisson process with rate $\lambda$.
	Each goal is made by team A with probability $p$ and team B with probability $1 - p$. 
	For $j > 1$, we say that the jth goal is a turnaround if it is made by a different team than the $(j - 1)$st goal. 
	For example, in the sequence AABBA, the 3rd and 5th goals are
	turnarounds.
	\begin{enumerate}
		\item In $n$ goals, what is the expected number of turnarounds?
		\item What is the expected time between turnarounds, in continuous time?
	\end{enumerate}
	
	
	
	\stepcounter{num} 
	\paragraph{Exercise \thenum.}
	Bees in a forest are distributed according to a 3D Poisson process with rate $\lambda$. 
	What is the distribution of the distance from a hiker to the nearest bee?
	
	

\stepcounter{num} 
\paragraph{Exercise \thenum.}
		Suppose cars enter a one-way highway from a common entrance, following a Poisson process with rate $\lambda$. The $i$-th car has velocity $V_i$ and travels at this velocity forever; no time is lost when one car overtakes another car. Assume the $V_i$ are i.i.d. discrete random variables whose support is a finite set of positive values. The process starts at time 0, and we'll consider the highway entrance to be at location 0.
		
		For fixed locations $a$ and $b$ on the highway with $0 < a < b$, let $Z_t$ be the number of cars located in the interval $[a, b]$ at time $t$. (For instance, on an interstate highway running west to east through the midwestern United States, $a$ could be Kansas City and $b$ could be St. Louis; then $Z_t$ would be the number of cars on the highway that are in the state of Missouri at time $t$.) Figure 13.6 illustrates the setup of the problem and the definition of $Z_t$.
		
		Assume $t$ is large enough that $t > \frac{b}{V_i}$ for all possible values of $V_i$. Show that $Z_t$ has a Poisson distribution with mean $\lambda(b - a) \mathbb{E}(V_i^{-1})$.
		
	
	\stepcounter{num} 
	\paragraph{Exercise \thenum.}
	Two students are independently performing independent Bernoulli trials. 
	For concreteness, assume that student A is flipping a 1dH coin with probability $p_1$ of Heads and student B is flipping a 5dhs coin with probability $p_2$ of Heads. 
	Let $X_1, X_2, \dots$ be student A's results and $Y_1, Y_2, \dots$ be student B's results, with $X_i \sim \text{Bern}(p_1)$ and $Y_j \sim \text{Bern}(p_2)$.
	
	Find the distribution and expected value of the first time at which they are simultaneously successful, i.e., the smallest $n$ such that $X_n = Y_n = 1$.
	
	
	\paragraph{References and acknowledgments:} Introduction to probability (J. Blitzstein and J. Huang) - Introduction to probability (D. Bertsekas and J.  Tsitsiklis).
\end{document}