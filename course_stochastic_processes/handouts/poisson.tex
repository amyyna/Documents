\documentclass[12pt,a4paper]{article}
\usepackage[latin1]{inputenc}
\usepackage[T1]{fontenc} 
\usepackage[francais, english]{babel} 
\usepackage{amsmath, amsthm}
\newtheorem{theorem}{Theorem}
\newtheorem{proposition}{Proposition}
\newtheorem{lemma}{Lemma}
\newtheorem*{proof*}{Proof}
\newtheorem{definition}{Definition}

\newtheorem{exercise}{Exercise}
\newtheorem{question}{Question}

\newtheorem{example}{Example}

\newtheorem{remark}{Remark}

\usepackage[a4paper,left=2cm, right=2cm,top=3cm,bottom=3cm]{geometry}
\usepackage{cancel}
\usepackage{bm}
\usepackage{amssymb,amsfonts}
\usepackage{mathrsfs}
\usepackage{color}
%\usepackage{hyperref}
\usepackage{dsfont}
\usepackage{graphicx}
\usepackage{algorithmicx}
\usepackage[ruled]{algorithm}
\usepackage{algpseudocode}
\usepackage{marginnote}
\newcommand{\Ptr}{\mathcal P^{\rm tr}}
\newcommand{\tr}{{\rm tr}}
\newcommand{\N}{\mathbb N}
\newcommand{\calN}{\mathcal N}
\newcommand{\bP}{\bold P}
\newcommand{\calK}{\mathcal K}
\newcommand{\calF}{\mathcal F}
\newcommand{\calH}{\mathcal H}
\newcommand{\calP}{\mathcal P}
\newcommand{\calC}{\mathcal C}
\newcommand\red[1]{\textcolor{red} {#1} }
\newcommand{\R}{\mathbb R}
\newcommand{\bX}{\bar X}
\newcommand{\Ktr}{\calK^{\rm tr}}
\title{ \bfseries \Huge {Handout 3: Poisson processes }} 
\date{Due date : }       
\vspace{-1cm}        
\newcounter{num}  % Create a new counter for paragraphs

\setcounter{num}{1}  % Start the paragraph counter at 1
\begin{document}
	\maketitle
	\paragraph{Exercise \thenum.}
	Passengers arrive at a bus stop according to a Poisson process with rate $\lambda$. The arrivals of buses are exactly $t$ minutes apart. 
	\begin{enumerate}
		\item 
	Show that on average, the sum of the waiting times of the riders on one of the buses is $\frac 1 2 \lambda t^2$.
	\end{enumerate}
	
	
	\stepcounter{num} 
	\paragraph{Exercise \thenum.}
	Suppose we have a couple of Poisson processes with rates $\lambda_1, \lambda_2$ respectively. 
	For an interval of length $ \delta$ small enough, give an approximation of $P((k_1,k_2);\delta)$ for $k_1,k_2\in\{1,2\}$.
	A student receives phone calls according to a Poisson process with rate $ \lambda$. 
	Unfortunately she has lost her cell phone charger. 
	The battery's remaining life is a random variable $T$ with mean $\mu$ and variance $\sigma^2$. Let $N(T)$ be the number of phone calls she receives before the battery dies.
	\begin{enumerate}
		\item Find $\rm E[N(T)]$ and ${\rm Var}[N(T)]$.
	\end{enumerate}
	
	\stepcounter{num} 
	\paragraph{Exercise \thenum.}
On a whatsApp question-ans-answer group, $N \sim {\rm Pois}(\lambda_1)$ questions will be posted tomorrow, with $\lambda_1$ measured in questions/day. Given $N$ , the post times are i.i.d. and
uniformly distributed over the day (a day begins and ends at midnight). 
When a question is posted, it takes an ${\rm Exp}(\lambda_2)$ amount of time (in days) for an answer to be posted,
independently of what happens with other questions.
\begin{enumerate}
	\item Find the probability that a question posted at a uniformly random time tomorrow will not yet have been answered by the end of that day.
\end{enumerate}
	
	
	\stepcounter{num} 
	\paragraph{Exercise \thenum.}
	In an endless football match, goals are scored according to a Poisson process with rate $\lambda$.
	Each goal is made by team A with probability $p$ and team B with probability $1 ? p$. 
	For $j > 1$, we say that the jth goal is a turnaround if it is made by a different team than the $(j ? 1)$st goal. 
	For example, in the sequence AABBA, the 3rd and 5th goals are
	turnarounds.
	\begin{enumerate}
		\item In $n$ goals, what is the expected number of turnarounds?
		\item What is the expected time between turnarounds, in continuous time?
	\end{enumerate}
	
	
	
	\paragraph{References and acknowledgments:} Introduction to probability (Blitzstein and Huang) 
\end{document}