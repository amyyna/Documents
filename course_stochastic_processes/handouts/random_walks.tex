\documentclass[twocolumn,12pt,a4paper]{article}
\usepackage[latin1]{inputenc}
\usepackage[T1]{fontenc} 
\usepackage[english]{babel} 

\pagestyle{empty} 


\usepackage{amsmath, amsthm}
\newtheorem{theorem}{Theorem}
\newtheorem{proposition}{Proposition}
\newtheorem{lemma}{Lemma}
\newtheorem*{proof*}{Proof}
\newtheorem{definition}{Definition}

\newtheorem{exercise}{Exercise}
\newtheorem{question}{Question}

\newtheorem{example}{Example}

\newtheorem{remark}{Remark}

\usepackage[a4paper,left=.8cm, right=.8cm,top=1.5cm,bottom=1.5cm]{geometry}
\setlength{\columnsep}{1.2cm}

\usepackage{cancel}
\usepackage{bm}
\usepackage{amssymb,amsfonts}
\usepackage{mathrsfs}
\usepackage{color}
%\usepackage{hyperref}
\usepackage{dsfont}
\usepackage{graphicx}
\usepackage{algorithmicx}
\usepackage[ruled]{algorithm}
\usepackage{algpseudocode}
\usepackage{marginnote}
\newcommand{\Ptr}{\mathcal P^{\rm tr}}
\newcommand{\tr}{{\rm tr}}
\newcommand{\N}{\mathbb N}
\newcommand{\calN}{\mathcal N}
\newcommand{\bP}{\bold P}
\newcommand{\calK}{\mathcal K}
\newcommand{\calF}{\mathcal F}
\newcommand{\calH}{\mathcal H}
\newcommand{\calP}{\mathcal P}
\newcommand{\calC}{\mathcal C}
\newcommand\red[1]{\textcolor{red} {#1} }
\newcommand{\R}{\mathbb R}
\newcommand{\bX}{\bar X}
\newcommand{\Ktr}{\calK^{\rm tr}}
\title{ \bfseries \Huge {Handout 2: Random walks }}    
\vspace{-4cm}        

\date{Due date : February $13^{th}$}       
\vspace{-4cm}        
\newcounter{num}  % Create a new counter for paragraphs
\begin{document}
	\maketitle
	\setcounter{num}{1}  % Start the paragraph counter at 1
	
	\thispagestyle{empty} 
	\paragraph{Brief recap}
	\begin{enumerate}
	\item 	
		\end{enumerate}


\paragraph{Exercise \thenum.}

A CC student leaves Riad 9 after class.
The student is walking along the Riads' walkable alley and is torn between making a trip to Palm's to chat with friends and going to the learning center to finish MCS123 homework. 
The student is $n$ steps from Palms' and $T-n$ steps from the learning center where $T \geq 3$ is the distance between Palms' and the
learning center, both located on opposite ends of the Riads' walkable alley. 
The student flips a coin at each step to decide which way to go. 
With probability $p$, the student takes 2 steps toward the restaurant and with probability $1-p$, they take 1 step toward the learning center. 
If the student is one or two steps away from Palms' and takes two steps toward it, then the student reaches Palms'. 
If they are one step away from the learning center and take a step in its direction, then they reaches the learning center. 
Let $X_n$ be the probability that the student reaches the
restaurant before the learning center. 
Assume that all coin tosses are mutually independent.
Find a recurrence for $X_n$. (You do not need to solve the recurrence)

%X_n = p X_{n+2} + (1-p)X_{n-1}


\stepcounter{num} 
\paragraph{Exercise \thenum.}
A particle moves $n$ steps on a number line. 
The particle starts at 0, and at each step it moves 1 unit to the right or to the left, with equal probabilities. 
Assume all steps are independent. 
Let $Z$ be the particle's position after $n$ steps. 
Find the PMF of $Z$. 
%If k is an integer between −n and n (inclusive) such that n+k is an even number, we get
%$$ P(Z = k) = \binom{n}{\frac{n+k}{2}}\left(\frac{1}{2}\right)^n$$
%otherwise $ P(Z = k) = 0$


\stepcounter{num} 
\paragraph{Exercise \thenum.}
(Gambler's ruin). Two gamblers, A and B, make a sequence of
1\$ bets. In each bet, gambler A has probability p of winning, and gambler B has probability $q = 1 - p$ of winning. Gambler A starts with i dollars and gambler B starts with $N - i$ dollars; the total wealth between the two remains constant since every time A loses a dollar, the dollar goes to B, and vice versa.
We can visualize this game as a random walk on the integers between 0 and N , where p is the probability of going to the right in a given step: imagine a person who starts at position i and, at each time step, moves one step to the right with probability p and one step to the left with probability $q = 1 - p$. 
The game ends when either A or B is ruined, i.e., when the random walk reaches 0 or $N$. 
What is the probability that A wins the game (walking away with all the money)?


\stepcounter{num} 
\paragraph{Exercise \thenum.}
An immortal drunk man wanders around randomly on the integers. 
He starts at the origin, and at each step he moves 1 unit to the right or 1 unit to the left, with equal probabilities, independently of all
his previous steps. 
Let $b$ be a googolplex (this is $10g$, where $g = 10^{100}$ is a googol).
\begin{enumerate}
	\item Find a simple expression for the probability that the immortal drunk visits $b$ before returning to the origin for the first time.
	\item Find the expected number of times that the immortal drunk visits $b$ before returning to the origin for the first time.
\end{enumerate}

\end{document}